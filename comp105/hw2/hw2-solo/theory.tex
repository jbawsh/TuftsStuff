\documentclass{book}
% generated by Madoko, version 0.9.2-beta
%mdk-data-line={1}

\usepackage[heading-base=1]{madoko}


\begin{document}
%mdk-data-line={9}
{}\mdTitle{OpSem Theory}
\mdSubtitle{COMP105 Fall 2015}
\mdAuthor{James McCants}{}{}{}
\mdMaketitle{}

%mdk-data-line={15}
\newcommand{\state}[1]{\langle #1 \rangle}
\newcommand{\inference}[2]{\dfrac{#1}{#2}}
\newcommand{\yields}[0]{\Downarrow}\mdHxx[id=sec-problem-16,label={1},toc={},data-line={20},caption={Problem 16},bookmark={Problem 16}]{%mdk-data-line={20}
{}Problem 16}\begin{mdP}[data-line={22}]%
%mdk-data-line={22}
{}Here are the standard ImpCore inferences rules for %mdk-data-line={22}
{}\mdCode[class={code,code1}]{VAR(x)}%mdk-data-line={22}
{}:%
\end{mdP}%
\begin{mdP}[class={indent},data-line={24}]%
%mdk-data-line={24}
{}\mdSpan[class={math-inline},elem={math-inline}]{$\inference{\mathid{x}\mdMathspace{1}\in \mathkw{dom}\mdMathspace{1}\rho}{\state{\mathkw{VAR}(\mathid{x}),\mdMathspace{1}\xi,\mdMathspace{1}\phi,\mdMathspace{1}\rho}\mdMathspace{1}\yields \state{\rho(\mathid{x}),\mdMathspace{1}\xi,\mdMathspace{1}\phi,\mdMathspace{1}\rho}}$}%mdk-data-line={26}
{}%
\end{mdP}%
\begin{mdP}[class={indent},data-line={28}]%
%mdk-data-line={28}
{}and%
\end{mdP}%
\begin{mdP}[class={indent},data-line={30}]%
%mdk-data-line={30}
{}\mdSpan[class={math-inline},elem={math-inline}]{$\inference{\mathid{x}\mdMathspace{1}\notin \mathkw{dom}\mdMathspace{2}\rho \\\mdMathspace{1}\mathid{x}\mdMathspace{1}\in \mathkw{dom}\mdMathspace{1}\xi}{\state{\mathkw{VAR}(\mathid{x}),\mdMathspace{1}\xi,\mdMathspace{1}\phi,\mdMathspace{1}\rho}\mdMathspace{1}\yields \state{\xi(\mathid{x}),\mdMathspace{1}\xi,\mdMathspace{1}\phi,\mdMathspace{1}\rho}}$}%mdk-data-line={32}
{}%
\end{mdP}%
\mdHxxx[id=sec-a-awk-like-semantics,label={1},toc={},data-line={34},caption={(a) Awk-like semantics},bookmark={(a) Awk-like semantics}]{%mdk-data-line={34}
{}(a) Awk-like semantics}\mdHxxxx[id=sec-add-varx-,label={1},data-line={36},caption={Add 'VAR(x)':}]{%mdk-data-line={36}
{}Add %mdk-data-line={36}
{}{\textquoteleft}VAR(x){\textquoteright}%mdk-data-line={36}
{}:}\begin{mdP}[data-line={38}]%
%mdk-data-line={38}
{}\mdSpan[class={math-inline},elem={math-inline}]{$\inference{\mathid{x}\mdMathspace{1}\notin \mathkw{dom}\mdMathspace{2}\rho \\\mdMathspace{1}\mathid{x}\mdMathspace{1}\notin \mathkw{dom}\mdMathspace{1}\xi}{\state{\mathkw{VAR}(\mathid{x}),\mdMathspace{1}\xi,\mdMathspace{1}\phi,\mdMathspace{1}\rho}\mdMathspace{1}\yields \state{0,\mdMathspace{1}\xi \prime (\mathid{x}\rightarrow0),\mdMathspace{1}\phi,\mdMathspace{1}\rho}}$}%mdk-data-line={40}
{}%
\end{mdP}%
\mdHxxxx[id=sec-add-setx-,label={2},data-line={43},caption={Add 'SET(x)':}]{%mdk-data-line={43}
{}Add %mdk-data-line={43}
{}{\textquoteleft}SET(x){\textquoteright}%mdk-data-line={43}
{}:}\begin{mdP}[data-line={45}]%
%mdk-data-line={45}
{}\mdSpan[class={math-inline},elem={math-inline}]{$\inference{\mathid{x}\mdMathspace{1}\notin \mathkw{dom}\mdMathspace{2}\rho \\\mdMathspace{1}\mathid{x}\mdMathspace{1}\notin \mathkw{dom}\mdMathspace{1}\xi \\\mdMathspace{7}{\state{\mathid{e},\mdMathspace{1}\xi,\mdMathspace{1}\phi,\mdMathspace{1}\rho}\mdMathspace{1}\yields \state{\mathid{e},\mdMathspace{1}\xi \prime,\mdMathspace{1}\phi,\mdMathspace{1}\rho \prime}}}{\state{\mathkw{SET}(\mathid{x},\mdMathspace{1}\mathid{e}),\mdMathspace{1}\xi,\mdMathspace{1}\phi,\mdMathspace{1}\rho}\mdMathspace{1}\yields \state{\mathid{v},\mdMathspace{1}\xi \prime (\mathid{x}\rightarrow \mathid{v}),\mdMathspace{1}\phi,\mdMathspace{1}\rho \prime}}$}%mdk-data-line={47}
{}%
\end{mdP}%
\mdHxxx[id=sec-b-icon-like-semantics,label={2},toc={},data-line={50},caption={(b) Icon-like semantics},bookmark={(b) Icon-like semantics}]{%mdk-data-line={50}
{}(b) Icon-like semantics}\mdHxxxx[id=sec-add-varx-,label={3},data-line={52},caption={Add 'VAR(x)':}]{%mdk-data-line={52}
{}Add %mdk-data-line={52}
{}{\textquoteleft}VAR(x){\textquoteright}%mdk-data-line={52}
{}:}\begin{mdP}[data-line={54}]%
%mdk-data-line={54}
{}\mdSpan[class={math-inline},elem={math-inline}]{$\inference{\mathid{x}\mdMathspace{1}\notin \mathkw{dom}\mdMathspace{2}\rho \\\mdMathspace{1}\mathid{x}\mdMathspace{1}\notin \mathkw{dom}\mdMathspace{1}\xi}{\state{\mathkw{VAR}(\mathid{x}),\mdMathspace{1}\xi,\mdMathspace{1}\phi,\mdMathspace{1}\rho}\mdMathspace{1}\yields \state{0,\mdMathspace{1}\xi,\mdMathspace{1}\phi,\mdMathspace{1}\rho \prime (\mathid{x}\rightarrow0)}}$}%mdk-data-line={56}
{}%
\end{mdP}%
\mdHxxxx[id=sec-add-setx-,label={4},data-line={58},caption={Add 'SET(x)':}]{%mdk-data-line={58}
{}Add %mdk-data-line={58}
{}{\textquoteleft}SET(x){\textquoteright}%mdk-data-line={58}
{}:}\begin{mdP}[data-line={60}]%
%mdk-data-line={60}
{}\mdSpan[class={math-inline},elem={math-inline}]{$\inference{\mathid{x}\mdMathspace{1}\notin \mathkw{dom}\mdMathspace{2}\rho \\\mdMathspace{1}\mathid{x}\mdMathspace{1}\notin \mathkw{dom}\mdMathspace{1}\xi \\\mdMathspace{8}{\state{\mathid{e},\mdMathspace{1}\xi,\mdMathspace{1}\phi,\mdMathspace{1}\rho}\mdMathspace{1}\yields \state{\mathid{e},\mdMathspace{1}\xi,\mdMathspace{1}\phi,\mdMathspace{1}\rho \prime}}}{\state{\mathkw{SET}(\mathid{x},\mdMathspace{1}\mathid{e}),\mdMathspace{1}\xi,\mdMathspace{1}\phi,\mdMathspace{1}\rho}\mdMathspace{1}\yields \state{\mathid{v},\mdMathspace{1}\xi,\mdMathspace{1}\phi,\mdMathspace{1}\rho \prime (\mathid{x}\rightarrow \mathid{v})}}$}%mdk-data-line={62}
{}%
\end{mdP}%
\mdHxxx[id=sec-c-which-do-you-prefer-and-why,label={3},toc={},data-line={64},caption={(c) Which do you prefer and why?},bookmark={(c) Which do you prefer and why?}]{%mdk-data-line={64}
{}(c) Which do you prefer and why?}\begin{mdP}[data-line={66}]%
%mdk-data-line={66}
{}I prefer the change to Icon because keeping variables that can be declared implicitly in a local environment seems safer. It limits the possibility to break things that rely on the global environment.%
\end{mdP}%
\mdHxx[id=sec-problem-13,label={2},toc={},data-line={68},caption={Problem 13},bookmark={Problem 13}]{%mdk-data-line={68}
{}Problem 13}\begin{mdP}[data-line={70}]%
%mdk-data-line={70}
{}\mdSpan[class={math-inline},elem={math-inline}]{$\inference{\inference{\inference{\mathid{x}\mdMathspace{1}\in \mathkw{dom}\mdMathspace{1}\rho \\\mdMathspace{1}\rho (\mathid{x})\mdMathspace{1}=\mdMathspace{1}99}{\state{\mathkw{VAR}(\mathid{x}),\mdMathspace{1}\xi,\mdMathspace{1}\phi,\mdMathspace{1}\rho}\mdMathspace{1}\yields \state{99,\mdMathspace{1}\xi,\mdMathspace{1}\phi,\mdMathspace{1}\rho}}\\\mdMathspace{1}\inference{}{\state{\mathkw{LITERAL}(3),\mdMathspace{1}\xi,\mdMathspace{1}\phi,\mdMathspace{1}\rho}\mdMathspace{1}\yields \state{3,\mdMathspace{1}\xi,\mdMathspace{1}\phi,\mdMathspace{1}\rho}}}{\state{\mathkw{SET}(\mathkw{VAR}(\mathid{x}),\mdMathspace{1}\mathkw{LITERAL}(3)),\mdMathspace{1}\xi,\mdMathspace{1}\phi,\mdMathspace{1}\rho}\mdMathspace{1}\yields \state{3,\mdMathspace{1}\xi \prime,\mdMathspace{1}\phi,\mdMathspace{1}\rho \prime (\mathid{x}\mdMathspace{1}\rightarrow 3)}}\\\mdMathspace{1}\inference{\mathid{x}\mdMathspace{1}\in \mathkw{dom}\mdMathspace{1}\rho \prime \\\mdMathspace{1}\rho \prime (\mathid{x})\mdMathspace{1}=\mdMathspace{1}3}{\state{\mathkw{VAR}(\mathid{x}),\mdMathspace{1}\xi \prime,\mdMathspace{1}\phi,\mdMathspace{1}\rho \prime}\mdMathspace{1}\yields \state{3,\mdMathspace{1}\xi \prime,\mdMathspace{1}\phi,\mdMathspace{1}\rho \prime}}}{\state{\mathkw{BEGIN}((\mathkw{SET},\mdMathspace{1}\mathkw{VAR}(\mathid{x}),\mdMathspace{1}\mathkw{LITERAL}(3))\mdMathspace{1}\mathkw{VAR}(\mathid{x})),\mdMathspace{1}\xi,\mdMathspace{1}\phi,\mdMathspace{1}\rho}\mdMathspace{1}\yields \state{3,\mdMathspace{1}\xi \prime,\mdMathspace{1}\phi,\mdMathspace{1}\rho \prime}}$}%mdk-data-line={84}
{}%
\end{mdP}%
\mdHxxx[id=sec-the-cut-off-line-,label={4},toc={},data-line={86},caption={The cut off line:},bookmark={The cut off line:}]{%mdk-data-line={86}
{}The cut off line:}\begin{mdP}[data-line={88}]%
%mdk-data-line={88}
{}\mdSpan[class={math-inline},elem={math-inline}]{$\state{\mathkw{VAR}(\mathid{x}),\mdMathspace{1}\xi \prime,\mdMathspace{1}\phi,\mdMathspace{1}\rho \prime}\mdMathspace{1}\yields \state{3,\mdMathspace{1}\xi \prime,\mdMathspace{1}\phi,\mdMathspace{1}\rho \prime}$}%mdk-data-line={88}
{}%
\end{mdP}%
\mdHxx[id=sec-problem-14,label={3},toc={},data-line={90},caption={Problem 14},bookmark={Problem 14}]{%mdk-data-line={90}
{}Problem 14}\mdHxxxx[id=sec-iftrue-,label={5},data-line={92},caption={IfTrue:}]{%mdk-data-line={92}
{}IfTrue:}\begin{mdP}[data-line={94}]%
%mdk-data-line={94}
{}\mdSpan[class={math-inline},elem={math-inline}]{$\inference{\state{\mathkw{VAR}(\mathid{x}),\mdMathspace{1}\xi,\mdMathspace{1}\phi,\mdMathspace{1}\rho}\mdMathspace{1}\yields \state{\mathid{v}_1,\mdMathspace{1}\xi,\mdMathspace{1}\phi,\mdMathspace{1}\rho}\\\mdMathspace{8}\mathid{v}_1\mdMathspace{1}\neq 0\mdMathspace{1}\\\mdMathspace{8}\state{\mathkw{VAR}(\mathid{x}),\mdMathspace{1}\xi,\mdMathspace{1}\phi,\mdMathspace{1}\rho}\mdMathspace{1}\yields \state{\mathid{v}_2,\mdMathspace{1}\xi \prime \prime,\mdMathspace{1}\phi,\mdMathspace{1}\rho \prime \prime}}{\state{\mathkw{IF}(\mathkw{VAR}(\mathid{x}),\mdMathspace{1}\mathkw{VAR}(\mathid{x}),\mdMathspace{1}\mathkw{LITERAL}(0)),\mdMathspace{1}\xi,\mdMathspace{1}\phi,\mdMathspace{1}\rho}\mdMathspace{1}\yields \state{\mathid{v}_2,\mdMathspace{1}\xi \prime \prime,\mdMathspace{1}\phi,\mdMathspace{1}\rho \prime \prime}}$}%mdk-data-line={97}
{}%
\end{mdP}%
\mdHxxxx[id=sec-in-this-case--v_1-v_2-neq-0,label={6},data-line={99},caption={In this case:               \$v\_1 = v\_2 {\textbackslash}neq 0\$}]{%mdk-data-line={99}
{}In this case:               %mdk-data-line={99}
{}\mdSpan[class={math-inline},elem={math-inline}]{$\mathid{v}_1\mdMathspace{1}=\mdMathspace{1}\mathid{v}_2\mdMathspace{1}\neq 0$}}\mdHxxxx[id=sec-iffalse-,label={7},data-line={101},caption={IfFalse:}]{%mdk-data-line={101}
{}IfFalse:}\begin{mdP}[data-line={103}]%
%mdk-data-line={103}
{}\mdSpan[class={math-inline},elem={math-inline}]{$\inference{\state{\mathkw{VAR}(\mathid{x}),\mdMathspace{1}\xi,\mdMathspace{1}\phi,\mdMathspace{1}\rho}\mdMathspace{1}\yields \state{\mathid{v}_1,\mdMathspace{1}\xi,\mdMathspace{1}\phi,\mdMathspace{1}\rho}\\\mdMathspace{8}\mathid{v}_1\mdMathspace{1}=\mdMathspace{1}0\mdMathspace{1}\\\mdMathspace{8}\state{\mathkw{LITERAL}(0),\mdMathspace{1}\xi,\mdMathspace{1}\phi,\mdMathspace{1}\rho}\mdMathspace{1}\yields \state{\mathid{v}_2,\mdMathspace{1}\xi \prime \prime,\mdMathspace{1}\phi,\mdMathspace{1}\rho \prime \prime}}{\state{\mathkw{IF}(\mathkw{VAR}(\mathid{x}),\mdMathspace{1}\mathkw{VAR}(\mathid{x}),\mdMathspace{1}\mathkw{LITERAL}(0)),\mdMathspace{1}\xi,\mdMathspace{1}\phi,\mdMathspace{1}\rho}\mdMathspace{1}\yields \state{\mathid{v}_2,\mdMathspace{1}\xi \prime \prime,\mdMathspace{1}\phi,\mdMathspace{1}\rho \prime \prime}}$}%mdk-data-line={106}
{}%
\end{mdP}%
\mdHxxxx[id=sec-in-this-case--v_1-v_2-0,label={8},data-line={108},caption={In this case:               \$v\_1 = v\_2 = 0\$}]{%mdk-data-line={108}
{}In this case:               %mdk-data-line={108}
{}\mdSpan[class={math-inline},elem={math-inline}]{$\mathid{v}_1\mdMathspace{1}=\mdMathspace{1}\mathid{v}_2\mdMathspace{1}=\mdMathspace{1}0$}}\mdHxx[id=sec-problem-23,label={4},toc={},data-line={110},caption={Problem 23},bookmark={Problem 23}]{%mdk-data-line={110}
{}Problem 23}\mdHxxx[id=sec-base-cases,label={5},toc={},data-line={112},caption={Base Cases},bookmark={Base Cases}]{%mdk-data-line={112}
{}Base Cases}\begin{mdP}[data-line={114}]%
%mdk-data-line={114}
{}Literal: a) In the case of Literal %mdk-data-line={114}
{}\mdSpan[class={math-inline},elem={math-inline}]{$\rho$}%mdk-data-line={114}
{} is popped off the stack and pushed back on 
with no change. b) Because %mdk-data-line={115}
{}\mdSpan[class={math-inline},elem={math-inline}]{$\rho$}%mdk-data-line={115}
{} is not changed in any way nothing is thrown away. 
There is no possibility for the stack to be missing environments.%
\end{mdP}%
\begin{mdP}[class={indent},data-line={118}]%
%mdk-data-line={118}
{}FormalVar: a) In the case of FormalVar %mdk-data-line={118}
{}\mdSpan[class={math-inline},elem={math-inline}]{$\rho$}%mdk-data-line={118}
{} is popped off the stack, checked for x, 
and then pushed back on when x is found. b) Because %mdk-data-line={119}
{}\mdSpan[class={math-inline},elem={math-inline}]{$\rho$}%mdk-data-line={119}
{} is not changed in any way 
nothing is thrown away. There is no possibility for the stack to be missing environments
.%
\end{mdP}%
\begin{mdP}[class={indent},data-line={123}]%
%mdk-data-line={123}
{}GlobalVar: a) In the case of GlobalVar %mdk-data-line={123}
{}\mdSpan[class={math-inline},elem={math-inline}]{$\rho$}%mdk-data-line={123}
{} is popped off the stack, checked for x, 
and then pushed back on when x is not found. b) Because %mdk-data-line={124}
{}\mdSpan[class={math-inline},elem={math-inline}]{$\rho$}%mdk-data-line={124}
{} is not changed in any 
way nothing is thrown away. There is no possibility for the stack to be missing 
environments.%
\end{mdP}%
\begin{mdP}[class={indent},data-line={128}]%
%mdk-data-line={128}
{}EmptyBegin: a) In the case of EmptyBegin %mdk-data-line={128}
{}\mdSpan[class={math-inline},elem={math-inline}]{$\rho$}%mdk-data-line={128}
{} is popped off the stack and pushed back 
on with no change. b) Because %mdk-data-line={129}
{}\mdSpan[class={math-inline},elem={math-inline}]{$\rho$}%mdk-data-line={129}
{} is not changed in any way nothing is thrown away. 
There is no possibility for the stack to be missing environments.%
\end{mdP}%
\begin{mdP}[class={indent},data-line={132}]%
%mdk-data-line={132}
{}ApplyAdd: a) In the case of ApplyAdd %mdk-data-line={132}
{}\mdSpan[class={math-inline},elem={math-inline}]{$\rho$}%mdk-data-line={132}
{} is not used in the addition so it can stay 
on the stack. b) Because %mdk-data-line={133}
{}\mdSpan[class={math-inline},elem={math-inline}]{$\rho$}%mdk-data-line={133}
{} is not used it is not changed and nothing is thrown away
. There is no possibility for the stack to be missing environments.%
\end{mdP}%
\mdHxxx[id=sec-induction-steps,label={6},toc={},data-line={136},caption={Induction Steps},bookmark={Induction Steps}]{%mdk-data-line={136}
{}Induction Steps}\begin{mdP}[data-line={138}]%
%mdk-data-line={138}
{}FormalAssign: a) In the case of FormalAssign %mdk-data-line={138}
{}\mdSpan[class={math-inline},elem={math-inline}]{$\rho$}%mdk-data-line={138}
{} is popped off the stack and a 
recursive call to eval is made to find what x should be set to. Once x is modified to 
its new value %mdk-data-line={140}
{}\mdSpan[class={math-inline},elem={math-inline}]{$\rho \prime$}%mdk-data-line={140}
{}, which contains the updated x, is pushed onto the stack and 
the old %mdk-data-line={141}
{}\mdSpan[class={math-inline},elem={math-inline}]{$\rho$}%mdk-data-line={141}
{} is thrown away. b) No environments have been lost on the stack in this 
procedure because %mdk-data-line={142}
{}\mdSpan[class={math-inline},elem={math-inline}]{$\rho \prime$}%mdk-data-line={142}
{} contains the all of %mdk-data-line={142}
{}\mdSpan[class={math-inline},elem={math-inline}]{$\rho$}%mdk-data-line={142}
{} with just x updated.%
\end{mdP}%
\begin{mdP}[class={indent},data-line={144}]%
%mdk-data-line={144}
{}IfTrue: a) In the case of IfTrue %mdk-data-line={144}
{}\mdSpan[class={math-inline},elem={math-inline}]{$\rho$}%mdk-data-line={144}
{} is popped off the stack and a call to eval is 
made to determine the case of the if statement. Any change to %mdk-data-line={145}
{}\mdSpan[class={math-inline},elem={math-inline}]{$\rho$}%mdk-data-line={145}
{} results in the 
environment being copied with the change recorded. The updated environment is %mdk-data-line={146}
{}\mdSpan[class={math-inline},elem={math-inline}]{$\rho \prime$}%mdk-data-line={147}
{} and then a second call to eval is made to check what to do when true. Any 
changes here are recorded similarily in %mdk-data-line={148}
{}\mdSpan[class={math-inline},elem={math-inline}]{$\rho \prime \prime$}%mdk-data-line={148}
{} which is then pushed on 
the stack. b) At every step something changes %mdk-data-line={149}
{}\mdSpan[class={math-inline},elem={math-inline}]{$\rho$}%mdk-data-line={149}
{} is copied with the change recorded 
into some %mdk-data-line={150}
{}\mdSpan[class={math-inline},elem={math-inline}]{$\rho \prime$}%mdk-data-line={150}
{} and then %mdk-data-line={150}
{}\mdSpan[class={math-inline},elem={math-inline}]{$\rho$}%mdk-data-line={150}
{} is thrown out. There is no loss of environment 
on the stack.%
\end{mdP}%
\begin{mdP}[class={indent},data-line={153}]%
%mdk-data-line={153}
{}IfFalse: a) In the case of IfFalse %mdk-data-line={153}
{}\mdSpan[class={math-inline},elem={math-inline}]{$\rho$}%mdk-data-line={153}
{} is popped off the stack and a call to eval is 
made to determine the case of the if statement. Any change to %mdk-data-line={154}
{}\mdSpan[class={math-inline},elem={math-inline}]{$\rho$}%mdk-data-line={154}
{} results in the 
environment being copied with the change recorded. The updated environment is %mdk-data-line={155}
{}\mdSpan[class={math-inline},elem={math-inline}]{$\rho \prime$}%mdk-data-line={156}
{} and then a second call to eval is made to check what to do when false. Any 
changes here are recorded similarily in %mdk-data-line={157}
{}\mdSpan[class={math-inline},elem={math-inline}]{$\rho \prime \prime$}%mdk-data-line={157}
{} which is then pushed on 
the stack. b) At every step something changes %mdk-data-line={158}
{}\mdSpan[class={math-inline},elem={math-inline}]{$\rho$}%mdk-data-line={158}
{} is copied with the change recorded 
into some %mdk-data-line={159}
{}\mdSpan[class={math-inline},elem={math-inline}]{$\rho \prime$}%mdk-data-line={159}
{} and then %mdk-data-line={159}
{}\mdSpan[class={math-inline},elem={math-inline}]{$\rho$}%mdk-data-line={159}
{} is thrown out. There is no loss of environment 
on the stack.                                                                         %mdk-data-line={160}
{} %mdk-data-line={160}
{}%
\end{mdP}%
\begin{mdP}[class={indent},data-line={162}]%
%mdk-data-line={162}
{}WhileIterate: a) In the case of WhileIterate %mdk-data-line={162}
{}\mdSpan[class={math-inline},elem={math-inline}]{$\rho$}%mdk-data-line={162}
{} is popped off the stack and eval is 
called on %mdk-data-line={163}
{}\mdSpan[class={math-inline},elem={math-inline}]{$\mathid{e}_1$}%mdk-data-line={163}
{}, any change is recorded and creates %mdk-data-line={163}
{}\mdSpan[class={math-inline},elem={math-inline}]{$\rho \prime$}%mdk-data-line={163}
{}. If %mdk-data-line={163}
{}\mdSpan[class={math-inline},elem={math-inline}]{$\mathid{v}_1\mdMathspace{1}\neq 0$}%mdk-data-line={163}
{} then 
%mdk-data-line={164}
{}\mdSpan[class={math-inline},elem={math-inline}]{$\mathid{e}_2$}%mdk-data-line={164}
{} is evaluated and any change to %mdk-data-line={164}
{}\mdSpan[class={math-inline},elem={math-inline}]{$\rho \prime$}%mdk-data-line={164}
{} is recorded in %mdk-data-line={164}
{}\mdSpan[class={math-inline},elem={math-inline}]{$\rho \prime \prime$}%mdk-data-line={164}
{}. 
The whole thing then recursively calls itself starting with %mdk-data-line={165}
{}\mdSpan[class={math-inline},elem={math-inline}]{$\rho \prime \prime$}%mdk-data-line={165}
{} as the 
initial environment. b) %mdk-data-line={166}
{}\mdSpan[class={math-inline},elem={math-inline}]{$\rho \prime$}%mdk-data-line={166}
{} is not pushed back on the stack until WhileEnd so 
this is addressed there.%
\end{mdP}%
\begin{mdP}[class={indent},data-line={169}]%
%mdk-data-line={169}
{}WhileEnd: a) In the case of WhileEnd %mdk-data-line={169}
{}\mdSpan[class={math-inline},elem={math-inline}]{$\rho$}%mdk-data-line={169}
{} is popped off the stack and eval is called 
on %mdk-data-line={170}
{}\mdSpan[class={math-inline},elem={math-inline}]{$\mathid{e}_1$}%mdk-data-line={170}
{}, any change is recorded and creates %mdk-data-line={170}
{}\mdSpan[class={math-inline},elem={math-inline}]{$\rho \prime$}%mdk-data-line={170}
{}. If %mdk-data-line={170}
{}\mdSpan[class={math-inline},elem={math-inline}]{$\mathid{v}_1\mdMathspace{1}=\mdMathspace{1}0$}%mdk-data-line={170}
{} then %mdk-data-line={170}
{}\mdSpan[class={math-inline},elem={math-inline}]{$\rho \prime$}%mdk-data-line={171}
{} is pushed onto the stack. b) At every step something changes %mdk-data-line={171}
{}\mdSpan[class={math-inline},elem={math-inline}]{$\rho$}%mdk-data-line={171}
{} is copied 
with the change recorded into some %mdk-data-line={172}
{}\mdSpan[class={math-inline},elem={math-inline}]{$\rho \prime$}%mdk-data-line={172}
{} and then %mdk-data-line={172}
{}\mdSpan[class={math-inline},elem={math-inline}]{$\rho$}%mdk-data-line={172}
{} is thrown out. There 
is no loss of environment on the stack.                                               %mdk-data-line={173}
{} %mdk-data-line={173}
{}%
\end{mdP}%
\begin{mdP}[class={indent},data-line={175}]%
%mdk-data-line={175}
{}Begin: a) In the case of Begin, %mdk-data-line={175}
{}\mdSpan[class={math-inline},elem={math-inline}]{$\rho$}%mdk-data-line={175}
{} is popped off of the stack and then every %mdk-data-line={175}
{}\mdSpan[class={math-inline},elem={math-inline}]{$\mathid{e}_\mathid{n}$}%mdk-data-line={175}
{} 
is evaluated. Each evaluation results in %mdk-data-line={176}
{}\mdSpan[class={math-inline},elem={math-inline}]{$\rho_\mathid{n}$}%mdk-data-line={176}
{} being changed to %mdk-data-line={176}
{}\mdSpan[class={math-inline},elem={math-inline}]{$\rho_\mathid{n}\mdMathspace{1}\prime$}%mdk-data-line={176}
{}. 
Where the old environment is copied with the changes recorded. When the last %mdk-data-line={177}
{}\mdSpan[class={math-inline},elem={math-inline}]{$\mathid{e}_\mathid{n}$}%mdk-data-line={177}
{} has 
been evaluated then %mdk-data-line={178}
{}\mdSpan[class={math-inline},elem={math-inline}]{$\rho \prime$}%mdk-data-line={178}
{} is pushed onto the stack. b) At every step something 
changes %mdk-data-line={179}
{}\mdSpan[class={math-inline},elem={math-inline}]{$\rho$}%mdk-data-line={179}
{} is copied with the change recorded into some %mdk-data-line={179}
{}\mdSpan[class={math-inline},elem={math-inline}]{$\rho \prime$}%mdk-data-line={179}
{} and then 
%mdk-data-line={180}
{}\mdSpan[class={math-inline},elem={math-inline}]{$\rho$}%mdk-data-line={180}
{} is thrown out. There is no loss of environment on the stack.%
\end{mdP}%
\begin{mdP}[class={indent},data-line={182}]%
%mdk-data-line={182}
{}ApplyUser: a) In the case of ApplyUser, %mdk-data-line={182}
{}\mdSpan[class={math-inline},elem={math-inline}]{$\rho$}%mdk-data-line={182}
{} is popped off of the stack and then 
every %mdk-data-line={183}
{}\mdSpan[class={math-inline},elem={math-inline}]{$\mathid{e}_\mathid{n}$}%mdk-data-line={183}
{} is evaluated. Each evaluation results in %mdk-data-line={183}
{}\mdSpan[class={math-inline},elem={math-inline}]{$\rho_\mathid{n}$}%mdk-data-line={183}
{} being changed to %mdk-data-line={183}
{}\mdSpan[class={math-inline},elem={math-inline}]{$\rho_\mathid{n}\mdMathspace{1}\prime$}%mdk-data-line={184}
{}. Where the old environment is copied with the changes recorded. When the last 
%mdk-data-line={185}
{}\mdSpan[class={math-inline},elem={math-inline}]{$\mathid{e}_\mathid{n}$}%mdk-data-line={185}
{} has been evaluated then %mdk-data-line={185}
{}\mdSpan[class={math-inline},elem={math-inline}]{$\rho \prime$}%mdk-data-line={185}
{} is pushed onto the stack. b) At every step 
something changes %mdk-data-line={186}
{}\mdSpan[class={math-inline},elem={math-inline}]{$\rho$}%mdk-data-line={186}
{} is copied with the change recorded into some %mdk-data-line={186}
{}\mdSpan[class={math-inline},elem={math-inline}]{$\rho \prime$}%mdk-data-line={186}
{} and 
then %mdk-data-line={187}
{}\mdSpan[class={math-inline},elem={math-inline}]{$\rho$}%mdk-data-line={187}
{} is thrown out. There is no loss of environment on the stack.%
\end{mdP}%
\begin{mdDiv}[class={logomadoko,block},elem={logomadoko},text-align={right},font-size={xx-small},margin-top={4em},data-line={194}]%
%mdk-data-line={195}
{}Created with{\mdNbsp}\mdA{https://www.madoko.net}{}{Madoko.net}.%
\end{mdDiv}%


\end{document}
